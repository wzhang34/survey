
\section{Introduction}
\label{sec:intro}

Due to the advancement of multimodal sensors, today's digital contents are inherently multimedia, e.g., text, image, audio, video. The multimedia data of interest covers a wide spectrum, ranging from text, audio, image, click-through log, Web videos to surveillance videos. Visual content, i.e., images and videos, in particular, become a new way of communication among Internet users with the proliferation of sensor-rich mobile devices. Accelerated by tremendous increase in Internet bandwidth and storage space, multimedia data has been generated, published and spread explosively, becoming an indispensable part of today's big data. 

Such large-scale multimedia data has opened challenges and opportunities for intelligent multimedia analysis, e.g., management, retrieval, recognition, categorization, visualization and generation. Meanwhile, with recent advances in deep learning techniques, we are now able to boost the intelligence of multimedia analysis significantly and initiate new research directions to analyze multimedia content. For instance, convolutional neural networks have demonstrated high capability in image and video recognition; recurrent neural networks are widely exploited in modeling temporal dynamics in videos; and generative adversarial networks are capable of generating realistic images on demand. Therefore, deep learning for intelligent multimedia analysis is becoming an emerging research area in the field of multimedia and computer vision.

The goal of this survey paper is to review state-of-the-arts deep learning components and network architectures, to identify typical scenarios and challenges emerging in multimedia analysis, and to discuss real-world datasets and benchmarks for future directions. 


